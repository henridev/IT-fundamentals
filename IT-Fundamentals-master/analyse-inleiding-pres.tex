\section{Inleiding}
\frame{\tableofcontents[currentsection]}

\subsection{Het veld $\mathbb{R},+,.$}
\begin{frame}
\frametitle{Eigenschappen in $\mathbb{R},+,.$}
\pause
\begin{eigenschap}
Stel $x,y,z\in \mathbb{R}$
\pause
\begin{itemize}
\item<+-> $\mathbb{R}$ is gesloten voor $+$ en $\cdot$: $x+y \in \mathbb{R}$ en $x\cdot  y \in \mathbb{R}$.
\item<+-> commutatief: $x+y=y+x$ en $x\cdot  y=y \cdot  x$.
\item<+-> associatief: $(x+y)+z=x+(y+z)$ en $(x\cdot  y)\cdot z = x\cdot  (y\cdot z)$.
\item<+-> distributief: $x\cdot (y+z)=x\cdot  y+x\cdot z$.
\item<+-> neutraal element: $x+0=x$ en $x\cdot  1=x$.
\item<+-> invers element voor +: er bestaat een element $-x\in \mathbb{R}$ zodat $x+(-x)=0$.
\item<+-> invers element voor $\cdot$: voor elke $x\in \mathbb{R}_{0}$ bestaat er een element $x^{-1}$ zodat $x\cdot  x^{-1}=1$.
\end{itemize}
\pause
Besluit: $\mathbb{R},+,\cdot $ is een veld.
\end{eigenschap}
\end{frame}

\subsection{Machten}
\begin{frame}
\frametitle{Definitie van een macht}
\pause
\begin{definitie}
Stel $a \in \mathbb{R}_0^+ \setminus \{1\}$
\begin{itemize}
\item<+-> dan geldt voor alle $n \in \mathbb{N}_0$
      \pause
      \[\begin{array}{rcl}
        a^n & = & a\cdot a\cdot a \cdots a ~~~(n\geq 2) \\ \pause
        a^1 & = & a \\ \pause
        a^0 & = & 1 \\ \pause
        a^{-n} & = & \frac{1}{a^n} \\ \pause
        a^{-1} & = & \frac{1}{a} \pause
        \end{array}\]
\item<+-> dan geldt voor alle $p,q \in \mathbb{N}$ \pause
      \[\begin{array}{rcl}
        a^{\frac{p}{q}} & = & \sqrt[q]{a^p} ~~~(q\geq 2)\\ \pause
        a^{-\frac{p}{q}} & = & \frac{1}{a^{\frac{p}{q}}} ~~~(q\geq 2)
        \end{array}\]
\end{itemize}
\end{definitie}
\end{frame}

\begin{frame}
\frametitle{Rekenregels}
\pause
\begin{eigenschap}
Stel $a,b \in \mathbb{R}_0^+\setminus\{1\}$. \\
Voor alle $r,s \in \mathbb{Q}$ geldt 
\pause
\[\begin{array}{rcl}
  a^r\cdot a^s & = & a^{r+s} \\~\\  \pause
  \displaystyle \frac{a^r}{a^s} & = & a^{r-s} \\~\\ \pause
  (a^r)^s & = & a^{rs} \\~\\ \pause
  a^r\cdot b^r & = & (a\cdot b)^r \\~\\ \pause
  \displaystyle\frac{a^r}{b^r} & = & \displaystyle(\frac{a}{b})^r
  \end{array}\]
\end{eigenschap}
\end{frame}

\subsection{Logaritmen}
\begin{frame}
\frametitle{Definitie van een logaritme}
\pause
\begin{definitie}

\end{definitie}
\end{frame}

\begin{frame}
\frametitle{Rekenregels}
\pause
\begin{eigenschap}

\end{eigenschap}
\end{frame}

\subsection{Intervallen in $\mathbb{R}$}
\begin{frame}
\frametitle{Notaties}
\pause
\begin{itemize}
\item<+-> Open interval:
      \[ ]a,b[=\{x\in \mathbb{R} | a<x<b\} \subseteq \mathbb{R} \]
\item<+-> Gesloten interval:
      \[ [a,b]=\{x\in \mathbb{R} | a\leq x \leq b\} \subseteq \mathbb{R} \]
\item<+-> Halfopen interval:
      \[ \begin{array}{l}
         [a,b[=\{x\in \mathbb{R} | a \leq x<b\} \subseteq \mathbb{R} \\
         ]a,b]=\{x\in \mathbb{R} | a < x \leq b\} \subseteq \mathbb{R} 
         \end{array}\]
\item<+-> De verzameling van de re\"ele getallen ($\mathbb{R}$):
      \[ \mathbb{R} = ]-\infty,+\infty[ \]
\end{itemize}
\end{frame}


\subsection{Het begrip oneindig in de wiskunde}
\begin{frame}
\frametitle{De uitgebreide re\"ele rechte}
\pause
\begin{definitie}
De {\bf uitgebreide re\"ele rechte} $\overline{\mathbb{R}}$ is:\\~\\
\[ \overline{\mathbb{R}} = \mathbb{R} \cup  \{-\infty,+\infty\} = [-\infty,+\infty] \]~\\
met $-\infty < x < +\infty$ voor elke $x \in \mathbb{R}$.
\end{definitie}
\end{frame}

\begin{frame}
\frametitle{Rekenregels voor `+' en `-' in $\overline{\mathbb{R}}$}
\pause
\begin{eigenschap}
Voor elke $x \in \mathbb{R}$ geldt \pause
\[\begin{array}{rcl}
  +\infty + x & = & +\infty\\ \pause
  -\infty + x & = & -\infty\\ \pause
  +\infty + (+\infty) & = & +\infty\\ \pause
  -\infty + (-\infty) & = & -\infty \pause
  \end{array} \]
\end{eigenschap}
~\\
\pause
Let op! De volgende bewerkingen zijn niet gedefinieerd
\pause
\[\begin{array}{c}
  +\infty + (-\infty) \\ \pause
  -\infty + (+\infty) 
  \end{array} \]
\end{frame}

\begin{frame}
\frametitle{Rekenregels voor `$\times$' en `$/$' in $\overline{\mathbb{R}}$}
\pause
\begin{eigenschap}
Voor elke $x \in \mathbb{R}_0^+$ geldt \pause
\[\begin{array}{rcl}
  +\infty \cdot  x & = & +\infty\\ \pause
  -\infty \cdot  x & = & -\infty\\ \pause
  +\infty / x & = & +\infty\\ \pause
  -\infty / x & = & -\infty   \pause
  \end{array} \]
\end{eigenschap}
\end{frame}

\begin{frame}
\frametitle{Rekenregels voor `$\times$' en `$/$' in $\overline{\mathbb{R}}$}
\begin{eigenschap}
Voor elke $x \in \mathbb{R}_0^-$ geldt \pause
\[\begin{array}{rcl}
  +\infty \cdot  x & = & -\infty\\ \pause
  -\infty \cdot  x & = & +\infty\\ \pause
  +\infty / x & = & -\infty\\ \pause
  -\infty / x & = & +\infty  \pause
  \end{array} \] 
\end{eigenschap}
\end{frame}

\begin{frame}
\frametitle{Rekenregels voor `$\times$' en `$/$' in $\overline{\mathbb{R}}$}
\begin{eigenschap}
Voor elke $x \in \mathbb{R}$ geldt \pause
\[\begin{array}{rcl}
  x/+\infty  & = & 0\\ \pause
  x/-\infty  & = & 0   
  \end{array} \]  
 \end{eigenschap}
\end{frame}

\begin{frame}
\frametitle{Rekenregels voor `$\times$' en `$/$' in $\overline{\mathbb{R}}$}
\begin{eigenschap}
Er geldt \pause
\[\begin{array}{rcl}
  +\infty \cdot  (+\infty) & = & +\infty\\ \pause
  -\infty \cdot  (-\infty) & = & +\infty\\ \pause
  +\infty \cdot  (-\infty) & = & -\infty\\ \pause
  -\infty \cdot  (+\infty) & = & -\infty  \pause
  \end{array} \]
\end{eigenschap}
 \pause
Let op! De volgende bewerkingen zijn niet gedefinieerd
\pause
\[\begin{array}{l}
  0\cdot  (+\infty)\\ \pause
  0\cdot  (-\infty)\\  \pause
  \infty / \infty \\ \pause
  0 / 0 \\  \pause
  0^0\\ \pause
  \infty^0  
  \end{array} \]
 \end{frame}
 
 