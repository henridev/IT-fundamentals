\section{Bijzondere functies}
\frame{\tableofcontents[currentsection]}

\subsection{De absolute waarde functie}
\begin{frame}
\frametitle{Definitie en eigenschappen}
\pause
\begin{definitie}
De {\bf absolute waarde functie} is de functie met als functievoorschrift
\[\mbox{abs}:\mathbb{R}\rightarrow \mathbb{R}:x\mapsto \sqrt{x^2}=\left\{\begin{array}{rr} x& ~~~~~~ \mbox{als } x\geq 0\\-x&\mbox{als } x<0 \end{array}\right.\]
\end{definitie}
%\end{frame}
~\\
%\begin{frame}
%\frametitle{Eigenschappen}
\pause
\begin{eigenschap}
\begin{itemize}
\item<+-> $\mbox{dom(abs)}=\mathbb{R}$
\item<+-> $\mbox{bld(abs)}=\mathbb{R}^{+}$
\item<+-> de functie abs vertoont een knik in het punt $(0,0)$ (cfr.\ singulier punt)
\end{itemize}
\end{eigenschap}
\end{frame}

\subsection{{\em Floor\/} en {\em Ceiling\/} functie}
\begin{frame}
\frametitle{De functie {\em Floor\/}}
\pause
\begin{definitie}
\[\mbox{floor}:\mathbb{R}\rightarrow \mathbb{Z}:x\mapsto \mbox{floor}(x)=z\] 
$\mbox{ met } z \mbox{ het grootste geheel getal zodat } z\leq x$
\end{definitie}
%\end{frame}
~\\
%\begin{frame}
%\frametitle{De functie {\em Floor\/}: eigenschappen}
\pause
\begin{eigenschap}
\pause
\begin{itemize}
\item<+-> $\mbox{dom(floor)}=\mathbb{R}$
\item<+->$\mbox{bld(floor)}=\mathbb{Z}$.
\end{itemize}
\end{eigenschap}
\end{frame}

\begin{frame}
\frametitle{De functie {\em Ceiling\/}}
\pause
\begin{definitie}
\[\mbox{ceiling}:\mathbb{R}\rightarrow \mathbb{Z}:x\mapsto \mbox{ceiling}(x)=z\]
$\mbox{ met } z \mbox{ het kleinste geheel getal zodat } z\geq x$
\end{definitie}
%\end{frame}
~\\
%\begin{frame}
%\frametitle{De functie {\em Ceiling\/}: eigenschappen}
\pause
\begin{eigenschap}
\begin{itemize}
\item<+-> dom(ceiling)$=\mathbb{R}$
\item<+-> bld(ceiling)$=\mathbb{Z}$
\end{itemize}
\end{eigenschap}
\end{frame}

\subsection*{Oefeningen}
\begin{frame}
\frametitle{Oefeningen}
\pause
\begin{enumerate}
\item[1]<+-> Stel $f:\mathbb{R}\rightarrow \mathbb{R}: x\mapsto \mbox{abs}(x+1)$.
      \begin{enumerate}
      \item[(a)] Geef het domein en beeld van $f$.
      \item[(b)] Bepaal alle nulpunten van $f$.
      \item[(c)] Schets de grafiek van $f$.
      \end{enumerate}
      ~\\
      \pause
\item[2]<+-> Stel 
\[\begin{array}{l}
  f:\mathbb{R}\rightarrow \mathbb{R}: x\mapsto \mbox{floor(ceiling}(x-\frac{1}{2}))\\
  g:\mathbb{R}\rightarrow \mathbb{R}: x\mapsto \mbox{floor(ceiling}(x)-\frac{1}{2})\\ 
  h:\mathbb{R}\rightarrow \mathbb{R}: x\mapsto \mbox{ceiling(floor}(x+\frac{1}{2})). 
	\end{array}\]
 Geef voor elk van de gegeven functies het domein en het beeld. Bepaal voor elke functie de nulpunten.
 Teken de grafiek van de functies $f$, $g$ en $h$.
\end{enumerate}
\end{frame}

