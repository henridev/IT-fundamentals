\section{Begrippen}
\frame{\tableofcontents[currentsection]}

\subsection{Definities en notaties}

\begin{frame}
\frametitle{Een re\"ele functie}
\pause
\begin{definitie} 
Een functie $f$ is een {\bfseries re\"ele functie} indien zijn bron- en doelverzameling beide $\mathbb{R}$ zijn. D.w.z.\ dat voor elke $x \in \mathbb{R}$ hoogstens \'e\'en $ y \in \mathbb{R}$ bestaat zodanig dat $f(x) =y $.\\
We noemen $x$ het {\bfseries argument} terwijl $y$ het {\bfseries beeld} wordt genoemd.
\end{definitie}
~\\
\pause
{\bfseries Notaties} ~\\
$f : \mathbb{R} \rightarrow \mathbb{R} : x \mapsto y$\\
${\mathcal{F}}(\mathbb{R},\mathbb{R}) =$ de verzameling van alle re\"ele functies.
\end{frame}

\begin{frame}
\frametitle{Het domein van een functie} 
\pause
\begin{definitie} 
Beschouw de functie $f:X\rightarrow Y:x\mapsto y=f(x)$.\\~
\pause

Het {\bfseries domein} of {\bfseries definitiegebied} van de functie $f$ is de verzameling van alle elementen $x\in X$ waarvoor $f(x) \in Y$:
\[\mbox{dom\,} f = \{x \in X \,|\, \mbox{er bestaat een } y \in Y \mbox{ zodat } y=f(x)\}.\]
\end{definitie}
\end{frame}

\begin{frame}
\frametitle{Het beeld van een functie} 
\pause
\begin{definitie}
	Beschouw de functie $f:X\rightarrow Y:x\mapsto y=f(x)$.\\~
	\pause
	
	Het {\bfseries beeld} van de functie $f$ is de verzameling van alle elementen $y$ waarvoor \mbox{$y=f(x)$} met $x \in X$:
	\[\mbox{bld\,} f = \{y \in Y \,|\, \mbox{er bestaat een } x \in X \mbox{ zodat } f(x)=y \}.\]
\end{definitie}
\end{frame}

\begin{frame}
\frametitle{Het begrip continu} 
\pause
\begin{definitie} Een intu\"{\i}tieve definitie:
	\pause
	\begin{itemize}
		\item<+-> Een re\"ele functie $f$ is {\bfseries continu}, indien we de grafiek van $f$ kunnen tekenen zonder ons potlood op te heffen. \\
		\item<+-> Een re\"ele functie $f$ is {\bfseries continu in een punt $a$} van haar domein, indien de grafiek van $f$ geen `sprong' vertoont in de onmiddellijke omgeving van het punt $(a,f(a))$. In het ander geval spreekt men van een {\bfseries discontinu\"{\i}teit in het punt $a$}. 
	\end{itemize}
\end{definitie}
\end{frame}

\begin{frame}
\frametitle{Een nulpunt van een functie $f$}
\pause
\begin{definitie}
Een {\bfseries nulpunt $x$} van een re\"ele functie $f$ is een element van het domein van $f$ waarvoor de functiewaarde gelijk is aan 0.\\ %snijpunt van de functie $f$ met de $X$-as (de rechte met vergelijking $y=0$).
M.a.w.\ een nulpunt $x$ van een functie $f$ is oplossing van de vergelijking $f(x)=0$.
\end{definitie}
\end{frame}

\begin{frame}
\frametitle{Het snijpunt met de $Y$-as van een functie $f$}
\pause
\begin{definitie}

\end{definitie}
\end{frame}

\begin{frame}
\frametitle{De tekentabel van een functie $f$}
\pause
\begin{definitie}
	
\end{definitie}
\end{frame}

\begin{frame}
\frametitle{Een asymptoot van een functie $f$}
\pause
\begin{definitie}
	
\end{definitie}
\end{frame}


